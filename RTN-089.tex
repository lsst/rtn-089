\documentclass[OPS,lsstdraft,authoryear,toc]{lsstdoc}
\input{meta}

% Package imports go here.

% Local commands go here.

%If you want glossaries
%\input{aglossary.tex}
%\makeglossaries

\title{Charge to the Survey Cadence Optimization Committee}

% This can write metadata into the PDF.
% Update keywords and author information as necessary.
\hypersetup{
    pdftitle={Charge to the Survey Cadence Optimization Committee},
    pdfauthor={Melissa Graham},
    pdfkeywords={}
}

% Optional subtitle
% \setDocSubtitle{A subtitle}

\author{%
Federica Bianco, Lynne Jones
}

\setDocRef{RTN-089}
\setDocUpstreamLocation{\url{https://github.com/lsst/rtn-089}}

\date{\vcsDate}

% Optional: name of the document's curator
% \setDocCurator{The Curator of this Document}

\setDocAbstract{%
This document provides the charge to the Rubin Observatory Legacy Survey of Space and Time (LSST) Survey Cadence Optimization Committee (SCOC).
}

% Change history defined here.
% Order: oldest first.
% Fields: VERSION, DATE, DESCRIPTION, OWNER NAME.
% See LPM-51 for version number policy.
\setDocChangeRecord{%
  \addtohist{1}{2024-10-04}{Updated and released as a Rubin tech note.}{Melissa Graham}
}


\begin{document}

% Create the title page.
\maketitle
% Frequently for a technote we do not want a title page  uncomment this to remove the title page and changelog.
% use \mkshorttitle to remove the extra pages

% ADD CONTENT HERE
% You can also use the \input command to include several content files.

\section{Purpose}

The SCOC is advisory to the Rubin Observatory operations director.
It began its work in 2020 and will continue throughout the operations
of Rubin Observatory and its 10-year Legacy Survey of Space and Time (LSST). 

The tasks of the SCOC are:

\begin{itemize}
\item Based on input from the Survey Schedule team and the LSST Science Collaborations, make specific recommendations for the cadence choices for the full 10-year survey. These recommendations will include a description of the pros and cons of the various choices, and will be in the form of one or more reports which will be made public.
\item Based on the plans for commissioning, and the realized performance of the telescope and software, make specific recommendations for “Early Science” observations, to be carried out at the end of commissioning and the first months of Rubin Observatory operations.
\item During operations, receive reports from the Survey Evaluation Working Group (SEWG), a project-internal group of scientists that will evaluate the current and expected performance of the survey and scheduler. Use this information, together with an understanding of the science outputs and changing scientific landscape of the Rubin Observatory, to make recommendations for changes in survey strategy, including Target of Opportunity observations and the use of Director’s Discretionary Time.
\item Help communicate these recommendations to the science community through, for example, posts on Rubin Community Forum and reports to the LSST Science Collaborations.

\end{itemize}


\subsection{Relationship to other committees}

The Science Advisory Committee (SAC) is responsible for drafting the SCOC charge and identifying
members to serve on the SCOC.

The Survey Evaluation Working Group (SEWG) is mostly comprised of Rubin staff members,
including representatives from Observatory Operations and Data Management.
On a quarterly basis, the SEWG evaluates the current and expected performance of the scheduler
and presents its findings both to the Operations Director and to the SCOC.


\section{Membership}

The SCOC will consist of roughly 10 individuals, named by the Science Advisory Committee (SAC)
and drawn almost entirely from the science community (i.e., from people not paid by the Rubin Observatory Project). 

While the existing LSST Science Collaborations will be a source of members of the SCOC, individuals
will not represent the specific scientific interests of the LSST Science Collaborations to which they belong.
Instead, like the SAC, members will work to optimize the global scientific productivity of the Rubin Observatory LSST.

The SCOC will be mostly comprised of scientists from both the US and Chile,
and should include at least one individual from Chile.

The SCOC will need expertise in the various areas which pull the cadence decisions in specific directions,
including:

\begin{itemize}
\item The inner and outer Solar System.
\item Low-latitude Milky Way science and crowded fields.
\item High-latitude Milky Way science, and resolved stars in nearby galaxies.
\item Variable stars and Milky Way transients.
\item Extragalactic transients.
\item Galaxies and strong lensing.
\item Active Galactic Nuclei (AGN).
\item Weak lensing cosmology.
\item Supernova cosmology.
\item Footprint and sky coverage with external surveys.
\item Data management.
\end{itemize}

To broaden the expertise of the SCOC itself, it will actively seek input and advice from members
of the scientific community and the LSST Science Collaborations.
The SCOC members will be free to share information presented in the SCOC discussions with the
experts with whom it consults.

Membership on this committee will initially be for two years, with possibility of renewal.


\subsection{Chair}

The SCOC will be chaired by the Deputy Project Scientist for Rubin Construction, who will be a non-voting member.

The SCOC chair is responsible for scheduling meetings (virtual and any in-person meetings) of the SCOC.

The SCOC chair is responsible for informing, and where appropriate consulting with, all relevant operations partners, including the US governmental funding agencies.



\section{Decision-Making Process}

The SCOC will start its deliberations in summer 2020.
It will meet by default once a month via telecon through the first year or so of operations.
It is anticipated that the rate of meetings can slow down after that.
Occasional face-to-face meetings will likely be necessary, especially as major decision points are reached.
The SCOC Chair will schedule the meetings.
The Project would provide logistical and financial support for those meetings
(including travel costs for all participants).
The Project will also endeavor to provide occasional travel support for SCOC members to travel
to scientific conferences to gain insight into some of the issues that the SCOC is exploring.

As mentioned above, the SCOC will reach out to the scientific community and the
LSST Science Collaborations when expertise is needed in specific areas.
An open process of communication with members of LSST Science Collaborations will
lead to better understanding of, and community buy-in to, the SCOC recommendations.

The SCOC would operate by consensus to the extent possible.
Formal votes might be taken at major decision points, such as:

\begin{itemize}
\item Development of an initial 10-year survey strategy plan, and drafting the report describing that plan.
\item Development of an “Early Science” plan.
\item Recommendations of significant changes in survey strategy during operations.
\end{itemize}

The results of these votes will be recorded in the SCOC minutes.


\section{Reports}

Notes from the SCOC meetings and the recommendation reports will be made public.
For example, by posting links to them in the Rubin Community Forum and at survey-strategy.lsst.io.



\appendix
% Include all the relevant bib files.
% https://lsst-texmf.lsst.io/lsstdoc.html#bibliographies
\section{References} \label{sec:bib}
\renewcommand{\refname}{} % Suppress default Bibliography section
\bibliography{local,lsst,lsst-dm,refs_ads,refs,books}

% Make sure lsst-texmf/bin/generateAcronyms.py is in your path
\section{Acronyms} \label{sec:acronyms}
\input{acronyms.tex}
% If you want glossary uncomment below -- comment out the two lines above
%\printglossaries





\end{document}
